\documentclass[11pt]{article}

\usepackage{polski}
\usepackage{float}
\usepackage{enumerate}
\usepackage{amsfonts}
\usepackage{indentfirst}
\usepackage{amsmath}
\usepackage{graphicx}
\usepackage{caption}
\usepackage{algorithm2e}
\usepackage[a4paper, total={6in, 8in}]{geometry}
\def \hfillx {\hspace*{-\textwidth} \hfill}
\graphicspath{ {./images/} }

\title{Algorytmy Metaheurystyczne\\\large Problem Komiwojażera}

\author{
        Szymon Brzeziński - 254611\\
        Paweł Prusisz - 254642
        }
\date{}

\begin{document}
\maketitle

\section{Opis}
Tematem pracy jest przetestowanie oraz opis niektórych zależności między algorytmami rozwiązującymi instancje problemu komiwojażera.
\\Badane instancje są wczytywane z biblioteki TSPLIB oraz generowane losowo.
\\Typy instancji:
\begin{enumerate}
    \item Symetryczne
    \item Asymetryczne
    \item Euklidesowe
\end{enumerate}
Badane algorytmy:
\begin{enumerate}
    \item k-random
    \item nearest neighbour
    \item extended nearest neighbour
    \item two-opt
\end{enumerate}
\section{Wyniki}
\subsection{Jakość rozwiązań }
Pierwszą badaną zależnością jest porównanie rozwiązań zwróconych przez algorytmy względem rozmiaru problemu. W tym celu dla każdego badanego rozmiaru  $n$ zostały wygenerowane $k$ różnych instancji. Długość zwróconej ścieżki oraz czas działania algorytmów został uśredniony dla każdego n.    

\end{document}